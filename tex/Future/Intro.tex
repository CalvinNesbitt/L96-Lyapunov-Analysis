\section{On Going Work}

\subsection{Angles Between the Lyapunov Vectors}

We have used the minimum principle angle between the stable and unstable manifolds to attempt to quantify violations of hyperbolicity, which we do indeed see. We are still working through some details to quantify how meaningful this is.

\subsection{Quantifying Rate Function Convergence}

There are two aspects of the rate function convergence we would like to quantify. The first is finding a meaningful time scale. We had in mind to use integrated autocorrelations as in \cite{Galfi2019}. The second we would like to quantify is some kind of cauchy convergence so that we have grounds for convergence beyond looking at the graphs.

\subsection{Effect of Time Scale Seperation}

We are currently in the process of varying the time scale seperation parameter $c$ as opposed to $h$ and carrying out the same analysis. A preliminary look at the LE spectrum seems to indicate this has a much larger effect.

\subsection{Increasing Certainty In Results}

We believe it will be worthwhile to calculate the Lyapunov exponents by using an algorithm other than that of Ginelli, perhaps to use the method of LU factorisation. The hope is that will help increase our certainty in what we have calculated. Causes for concern arise from \ref{fig:Error} and the disagreement between what has been calculated in \ref{fig:CLE} and \cite{Carlu2019}.
