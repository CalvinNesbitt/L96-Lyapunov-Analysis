\subsection{Large Deviation Principles}

Next we seek a large deviation principle for the FTLEs. In figure \ref{fig: h-Rate-Convergence} we have plotted a numerical estimate of the FTLE rate functions for increasing lengths of time $\tau$ (see section \ref{section: FTLE-Fluctuations}). Down the figure we plot rate functions for different indices of the Lyapunov exponents whilst across the figure we change the value of $c$.

% Rate Function Convergence Plot
\begin{figure}
    \centering
    \includegraphics[angle=90, height=\textwidth, keepaspectratio]{h-effect/Large-Comparison-Plot.pdf}
    \caption{Numerical estimate of the FTLE rate functions for increasing lengths of time $\tau$ (see section \ref{section: FTLE-Fluctuations}). Down the figure we plot rate functions for different indices of the Lyapunov exponents whilst across the figure we change the value of $c$.}
    \label{fig: h-Rate-Convergence}
\end{figure}

In figure \ref{fig: h-Rate-C-v-B} we have plotted a comparison of the FTCLE and FTBLE numerical rate functions. The time over which the FTLEs represent the growth rate is $100 \tau$ whilst the rate function is plotted in units of $\frac{1}{\lambda _1}$, which of course varies for each value of $c$. Down the figure we plot rate functions for different indices of the Lyapunov exponents whilst across the figure we change the value of $c$.

% Rate Function C v B Plot
\begin{figure}
    \centering
    \includegraphics[angle=90, width=\textwidth, keepaspectratio]{h-effect/Large-C-v-B-Plot.pdf}
    \caption{Comparison of the FTCLE and FTBLE numerical rate functions. The time over which the FTLEs represent the growth rate is $100 \tau$ whilst the rate function is plotted in units of $\frac{1}{\lambda _1}$, which of course varies for each value of $c$. Down the figure we plot rate functions for different indices of the Lyapunov exponents whilst across the figure we change the value of $c$.}
    \label{fig: h-Rate-C-v-B}
\end{figure}
