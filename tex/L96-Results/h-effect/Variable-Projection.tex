\subsection{Projection of CLV Variance on The Model Variables}

In figure \ref{fig:h-effect-projection} we plot the variance of each CLV by component. Note the first $36$ variable of the model are the synoptic variables whilst the next $360$ correspond to convective variables. We note that for all values of $c$ the all CLVs project on to the convective variables. Whilst it is only the near $0$ CLVs that project on to the synoptic variables. We also find that these CLVs project less on the convective variables in all cases. Finally, we note that increasing $c$ decreases the number of `near $0$` CLVs and hence reduces this band width.

% CLV Variance Projection
\begin{figure}
\centering
\includegraphics[width=\textwidth, keepaspectratio]{h-effect/CLV-Component-Projection}
\caption{Variance of each CLV by component. The y axis corresponds to the CLV index, whilst the x axis corresponds to the CLV component. The component variance is plotted with a logarithmic scale. The different panels correspond to increasing values of $c$ when read from the top left. Note the first $36$ variable of the model are the synoptic variables whilst the next $360$ correspond to convective variables. We note that for all values of $c$ the all CLVs project on to the convective variables. Whilst it is only the near $0$ CLVs that project on to the synoptic variables. We also find that these CLVs project less on the convective variables in all cases. Finally, we note that increasing $c$ decreases the number of `near $0$' CLVs and hence reduces this band width.}
\label{fig:h-effect-projection}
\end{figure}
