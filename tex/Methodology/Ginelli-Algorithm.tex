\subsection{The Ginelli Algorithm} \label{section: Ginelli-Alg}

The dynamics of a perturbation $\delta u$ are governed by the linearised dynamics of \ref{eqn: Dyn-Sys} often called the \textbf{tangent linear equation}:

\begin{align}
\dv{\delta u}{t} = J(u, t) \delta u \label{tangent-linear-eqn}
\end{align}

Following \cite{Ginelli2007} the way in which we have found the LEs and LVs is by solving \ref{eqn: Dyn-Sys} and \ref{tangent-linear-eqn} simultaneously whilst keeping track of local rates of growth in different directions.\\

To be more specific BLVs and the corresponding finite time backward Lyapunov exponents (FTBLEs) are found by implementing the so called Benettin steps:

\newcounter{algo}

\begin{enumerate}
\item We integrate the matrix equation corresponding to \ref{tangent-linear-eqn} $\dv{}{t} \vb{M}(t) = \vb{J}(\vb{u}) \vb{M}(t_0)$ numerically, for some time $\tau$ with initial condition $\epsilon \vb{I} _3$. This gives us $\vb{M}(\tau)$. Note that we have to integrate \ref{eqn: Dyn-Sys} simultaneously in order to evolve the jacobian.
\item We perform a $QR$ decomposition $\vb{M}(\tau) = \vb{QR}$.
\item We repeat the above with the $Q$ matrices now providing the initial condition. Eventually these $Q$'s will converge to the BLVs.
\item Moreover, at each $QR$ step we store the FTBLEs as $\tilde{\lambda}_i ^B (t_{k \tau})= \frac{1}{\tau} \log(\vb{R}_{ii})$ for $i = 1, 2, \ldots n$.
\item At the end of the process we estimate the LEs as the mean value of the FTBLE time series.
\setcounter{algo}{\value{enumi}}
\end{enumerate}

From here the basic idea to find the CLVs is to make use of the following:

\begin{enumerate}
\setcounter{enumi}{\value{algo}}
\item We can expand the CLVs in a basis of BLVs, $\vb{\Gamma} = \vb{Q A^-}$.
\item The stored $\vb{R}$s from the Benettin steps will map the coefficient matrices $\vb{A}^-$ to one another (with some stretching), i.e. $\vb{R A}^- (t_{n \tau}) = \vb{C A}^- (t_{(n+1) \tau})$, where $\vb{C}$ is a diagonal matrix.
\item If one keeps pushing the $\vb{A}^-$'s with stored $\vb{R}$s they will eventually converge to the real coefficient matrices. Since we have the BLVs we can then recover the CLVs.
\item The FTCLEs are then given by $\tilde{\lambda}_i ^C = \frac{1}{\tau} \log(\vb{C}_{ii})$ for $i = 1, 2, \ldots n$. As with the FTBLEs, their average values give the LEs.
\end{enumerate}
