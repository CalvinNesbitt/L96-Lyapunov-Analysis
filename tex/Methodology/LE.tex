\subsection{Basics of the Mathematical Theory of Lyapunov Exponents}

We will work with a dynamical system that is the flow determined by the ODE

\begin{align}
\dv{u}{t} = g(u). \label{eqn: Dyn-Sys}
\end{align}

The intuitive idea of Lyapunov exponents (LE) and their corresponding vectors is that at each point along the trajectory of \ref{eqn: Dyn-Sys} we can distinguish directions along which a perturbation will either grow or contract. The LE is then the exponential rate of expansion or contraction. The direction along which we perturb would be the corresponding Lyapunov vector (LV).\\

Formally LEs are defined via the multiplicative theorem of \cite{Oseledets1968}:

\begin{thm}[The Multiplicative Ergodic Theorem] \label{Oseldets}
Let $\mu$ be $f$ invariant borel measure. Then for a.e. $x$ you have $\lambda_1(x), \dots, \lambda_{n(x)}(x)$ with respective multiplicities $m_1, \dots, m_n$ such that:
\begin{enumerate}
\item For every tangent vector $v$, $\lambda(x,v) = \lambda_i$ for some $i$.\
\item $\dim(M) = \sum m_i$.\
\item $\sum m_i(x)\lambda(x) = \lim_{n \to \infty} \frac{1}{n} \log \abs{Df^n_x}$.
\end{enumerate}
\end{thm}

If $f$ is a diffeomorphism (invertible) you can decompose tangent space $T_x M$:

\begin{eqnarray}
T_x M = E_1 \oplus \dots \oplus E_n
\end{eqnarray}

where $\dim E_i = m_i$ and $\lambda(x,v) = \lambda_i$ for $v \in E_i$. The pairs $\{\lambda_i, m_i\}$ are the \textbf{Lyapunov exponents} along with their corresponding multiplicity. Note that \textbf{if your system is ergodic then LE are constant a.e.}.\\

Our goal is to find the $\lambda_i$ along with the corresponding tangent space partition $E_i$. This partition will be defined by a basis for the tangent space.\\

We will find two different set of basis vectors corresponding to the $E_i$: covariant Lyapunov vectors (CLVs) and backward Lyapunov vectors (BLVs). CLVs are covariant with the linearised dynamics whilst BLVs are orthogonal to one another.
